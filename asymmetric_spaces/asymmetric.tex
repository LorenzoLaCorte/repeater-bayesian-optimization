\documentclass{article}
\usepackage{amsmath}
\usepackage{hyperref}

\begin{document}

The enumeration of all possible binary trees with $N$ leaves is a fundamental topic in combinatorics and computer science. This problem is closely related to Catalan numbers, which arise in various counting problems involving recursive structures.

To find all possible binary trees with $N$ leaves, you can use the concept of Catalan numbers, which count the number of possible full binary trees. A full binary tree with $N$ leaves has $2N-1$ nodes. The number of distinct trees can be computed using dynamic programming, recursion, or combinatorial methods.

The number of full binary trees corresponds to the $N$-th Catalan number, given by:

\[
C_N = \frac{1}{N+1} \binom{2N}{N}
\]

This formula helps to determine the structure and count of possible trees with specific leaves. For more details, you can explore resources like Leetcode's problem on full binary trees\footnote{\url{https://leetcode.com/problems/all-possible-full-binary-trees/}}.

Here are some references that cover this topic in detail:

\begin{itemize}
    \item \textbf{Introduction to Algorithms} \\
    Thomas H. Cormen, Charles E. Leiserson, Ronald L. Rivest, and Clifford Stein \\
    \textit{Description:} A comprehensive textbook that covers fundamental algorithms and data structures, including binary trees. It discusses tree structures, traversal algorithms, and touches upon the enumeration of different binary trees. \\
    \textit{Edition:} 3rd Edition, MIT Press, 2009 \\
    \textit{ISBN:} 978-0262033848

    \item \textbf{Concrete Mathematics: A Foundation for Computer Science} \\
    Ronald L. Graham, Donald E. Knuth, and Oren Patashnik \\
    \textit{Description:} This book blends continuous and discrete mathematics and includes a thorough discussion of Catalan numbers and their applications in counting binary trees. \\
    \textit{Edition:} 2nd Edition, Addison-Wesley, 1994 \\
    \textit{ISBN:} 978-0201558029

    \item \textbf{Enumerative Combinatorics, Volume 2} \\
    Richard P. Stanley \\
    \textit{Description:} A detailed text on combinatorial enumeration, providing in-depth coverage of Catalan numbers, generating functions, and the enumeration of binary trees. \\
    \textit{Edition:} Cambridge University Press, 1999 \\
    \textit{ISBN:} 978-0521663519

    \item \textbf{Combinatorics: Topics, Techniques, Algorithms} \\
    Peter J. Cameron \\
    \textit{Description:} An introduction to combinatorial concepts, including tree enumeration and combinatorial structures related to binary trees. \\
    \textit{Edition:} Cambridge University Press, 1994 \\
    \textit{ISBN:} 978-0521457613
\end{itemize}

\textbf{Research Articles and Papers:}
\begin{itemize}
    \item \textit{The Number of Binary Trees with Integer Node Labels} \\
    Explores counting binary trees with various constraints.
    \item \textit{Catalan Numbers and Their Applications} \\
    An overview of Catalan numbers in combinatorial problems, including binary trees.
\end{itemize}

\textbf{Online Resources:}
\begin{itemize}
    \item \textbf{The On-Line Encyclopedia of Integer Sequences (OEIS):} \\
    Sequence A000108: Catalan Numbers \\
    Provides information on Catalan numbers with references to their occurrence in counting binary trees.
    \item \textbf{Wikipedia:}
    \begin{itemize}
        \item \href{https://en.wikipedia.org/wiki/Catalan_number}{Catalan Number} \\
        Offers a comprehensive overview, including formulas and applications to binary trees.
        \item \href{https://en.wikipedia.org/wiki/Binary_tree}{Binary Tree} \\
        Discusses the properties of binary trees and includes sections on counting and enumeration.
    \end{itemize}
    \item \textbf{MathWorld:} \\
    \href{https://mathworld.wolfram.com/CatalanNumber.html}{Catalan Number} \\
    A resource for mathematical definitions and properties related to Catalan numbers and binary trees.
\end{itemize}

\textbf{Generatingfunctionology} \\
Herbert S. Wilf \\
\textit{Description:} Focuses on generating functions as a method for counting combinatorial objects like binary trees. \\
\textit{Edition:} 3rd Edition, Academic Press, 2005 \\
\textit{ISBN:} 978-0127519565 \\
\textit{Availability:} Downloadable PDF provided by the author.

\textbf{Lecture Notes and Course Materials:}
\begin{itemize}
    \item Many universities provide lecture notes on combinatorics and data structures that include sections on binary tree enumeration. For example:
    \item \textbf{MIT OpenCourseWare:} \\
    Course 6.042J Mathematics for Computer Science \\
    Lecture Notes covering combinatorial structures.
\end{itemize}

These references should give you a solid foundation in understanding how to enumerate all possible binary trees with $N$ leaves. They cover theoretical aspects, provide mathematical proofs, and include examples that illustrate the concepts.

% TODO: Solve leetcode problem on full binary trees.
% TODO: Porting of the problem to our scenario, implementation and report here.
% TODO: Merge with main document.

\end{document}