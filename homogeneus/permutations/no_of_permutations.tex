\documentclass{article}
\usepackage{amsmath}

\begin{document}

\section*{Deriving the Number of Permutations}

To determine the number of possible permutations given parameters \(s\) and \(d\), where \(s\) represents the number of swaps and \(d\) represents the number of distillations, follow these steps:

\subsection*{Understanding the Problem}

Each permutation represents a sequence of binary digits (0s and 1s) where:
\begin{itemize}
    \item There are \(s\) zeros (swaps).
    \item There are \(d\) ones (distillations).
    \item The total length of the sequence is \(s + d\) because each position in the sequence is either a swap or a distillation.
\end{itemize}

\subsection*{Calculating Permutations}

To find the number of possible permutations, determine how many unique ways you can arrange \(s\) 1s in a sequence of length \(s + d\). This can be calculated using the binomial coefficient, which gives the number of ways to choose \(s\) positions out of \(s + d\) total positions.

The formula for the number of permutations is:
\[
\text{Number of permutations} = \binom{s+d}{s} = \frac{(s+d)!}{s! \cdot d!}
\]

where \(\binom{s+d}{s}\) is the binomial coefficient, and \((s+d)!\) is the factorial of \(s + d\).

\subsection*{Total number of permutations}

\[
\text{Number of permutations} = \sum_{s=1}^{max swaps} \sum_{d=0}^{max dists} \binom{s+d}{s}
\]


\subsection*{Example}

For 1 swap and a maximum of 3 distillations, the number of permutations is $10$, the sum of the following permutations:
\\
\begin{center}
\begin{tabular}{|c|c|c|}
\hline
Distillations & Numbers                      & Count \\
\hline
0             & 0                            & 1     \\
\hline
1             & 01, 10                       & 2     \\
\hline
2             & 011, 101, 110                & 3     \\
\hline
3             & 0111, 1011, 1101, 1110       & 4     \\
\hline
\end{tabular}
\end{center}

For 2 swaps and a maximum of 3 distillations the number of permutations is $20$, the sum of the following permutations:
\\
\begin{center}
\begin{tabular}{|c|c|c|}
\hline
Distillations & Numbers                                    & Count \\
\hline
0             & 00                                         & 1     \\
\hline
1             & 001, 010, 100                              & 3     \\
\hline
2             & 0011, 0101, 0110, 1001, 1010, 1100         & 6     \\
\hline
3             & 00111, 01011, 01101, 01110,                &       \\
              & 10011, 10101, 10110, 11001, 11010, 11100   & 10    \\
\hline
\end{tabular}
\end{center}

\subsubsection*{Big-O}

...

\end{document}
